\documentclass[11pt]{article}

\usepackage{url}

\newenvironment{list2}{
  \begin{list}{$\bullet$}{%
      \setlength{\itemsep}{0in}
      \setlength{\parsep}{0in} \setlength{\parskip}{0in}
      \setlength{\topsep}{0in} \setlength{\partopsep}{0in} 
      \setlength{\leftmargin}{0.5in}}}{\end{list}}\newenvironment{list1}{
  \begin{list}{\ding{113}}{%
      \setlength{\itemsep}{0in}
      \setlength{\parsep}{0in} \setlength{\parskip}{0in}
      \setlength{\topsep}{0in} \setlength{\partopsep}{0in} 
      \setlength{\leftmargin}{0.17in}}}{\end{list}}


\def\fullline{		% hrules only listen to \hoffset
  \nointerlineskip	% so I have this code	  
  \moveleft\hoffset\vbox{\hrule width\textwidth} 
  \nointerlineskip
}

\renewcommand{\rmdefault}{phv} % arial, uncomment to use times new roman
\renewcommand{\sfdefault}{phv} % arail, uncomment to use times new roman
\oddsidemargin=0.0in
\textwidth=6.0in

\begin{document}

\section{Elementary Notions of Heterogeneous Computing}

\subsection{Description}
This module introduces fundamental concepts in heterogeneous computing. Notions of concurrency,
parallelism, and energy efficiency are discussed to explain the motivation  behind the move towards
heterogeneous processing. Different forms of heterogeneity are introduced including soft
heterogeneity (i.e., difference in core compute capabilities within a multicore system), CPU-GPU
heterogeneous execution and System-on-Chip (SoC) design. The module also covers heterogeneity 
in workload and data with examples from cloud computing and mobile applications. The module concludes
with a discussion of programmability and performance challenges.

%, including the implications of
%Amdahl's Law for HC. 

\subsection{Context}
This module is primarily intended for CS1/CS2 students. Although the module introduces parallel
computing concepts before moving on to processor heterogeneity, it is ideally suited for a course with some
coverage of parallel computing material. For example, a CS1 course that incorporates a PDC module from
~\cite{CSINPARALLEL},~\cite{NSFCDER}, or ~\cite{TUESWEB}. In the absence of PDC coverage, the length
of this module will need to be increased or it will need to be combined with a PDC module.

\subsection{Topics}
The HC topics covered in this module are listed below. Bloom's classification is shown in brackets
\begin{list2}
\item Concurrency and Parallelism [K]
\item Multicore Processors [K]
\item GPGPU [K]
\item System-on-Chip (e.g., mobile processors)
\item Energy Efficiency [K]
\item Tasks and Workloads [K]
\item Task Mapping and Scheduling [K]
\item Amdahl's Law [C]
\end{list2}

\subsection{Learning Outcomes}
Having completed this module, students should be able to 
\begin{list2}
\item describe the differences between a homogeneous and heterogeneous computing system
\item describe and distinguish between different forms of heterogeneity
\item understand the motivation behind the design of heterogeneous computing systems
\item recognize the importance of energy efficiency on current computing systems
\item understand that tasks in a workload have different demands for compute and memory resources
\item understand the notion of task mapping as performed by an operating system
\item analyze the performance and energy effects of task mapping on a heterogeneous system
\end{list2}

\subsection{Pedagogical Notes}

To be updated. 

\subsection{Instructor Resources}
The teaching material included with this module include the following
\begin{list2}
\item Lecture slides: includes notes on guiding class discussion
\item In-class Demo: includes instructions for setting up a heterogeneous environment within a
  homogeneous multicore system and step-by-step guidelines for running the demo in class 
\item Lab : The module includes a lab that provides students hands-on experience in running
  application on a heterogeneous system. The lab will also reinforce the performance and energy
  implications covered in the lecture. The lab includes detailed instructions for the instructor in
  setting up a heterogeneous system on which students will conduct performance experiments. The lab
  requires the student to have some basic familiarity with a Linux environment.  
\item Reference Material: further reading for instructors unfamiliar with topics covered in this
  module 
\item Pedagogical Notes: suggestions drawn from author's own experience in teaching this module 
\end{list2}

All material is available for download from https://github.com/TeachingUndergradsCHC/modules.git
under folder moduleA. 

\bibliography{edu}
\bibliographystyle{unsrt}
\end{document}       
